\documentclass[a4paper,12pt]{article}

%\title{Deutsch: Die Hausaufgaben}
%\date{Oktober 15, 2014}
%\author{D.Babula \\ Bury GmBH L\"ohne Deutschlang }
\makeatletter

\def\thickhrulefill{\leavevmode \leaders \hrule height 1pt\hfill \kern \z@}
\def\maketitle{%
    \null
    \thispagestyle{empty}%
    \vfill
    \begin{center}\leavevmode
    \normalfont
    {\LARGE\raggedleft \@author\par}%
    \thickhrulefill\par
    {\huge\raggedright \@title\par}%
    \vskip 1cm
    \end{center}%
    \vfill
    {\Large \@date\par}%
    \null
    \cleardoublepage
    }
    \makeatother
    \author{Dawid Babula}
    \title{Deutsch -- die Hausaufgaben}
    \date{Oktober 15, 2014}

\begin{document}
\maketitle

\begin{enumerate}
    \item K\"onnten Sie mir bitte den Namen Ihrer Frau buchstabieren?
    \item Der Name des ersten amerikanischen Pr\"asidenten war George.
    \item Die Gastfreundschaft der Polen ist weltweit ber\"umt.
    \item Heute,kommen meine Mutter mit meinem Neffen bei uns vorbei.
    \item Der Gr\"undungsmythos Krakaus ist die Sage von dem Krak,der den Drachen get\"otet hatte.
    \item Ich suche nach dem guten Blumenlieferanten,\"uber dem ich meiner Frau einen sch\"onen Strau{\ss} schicke.
    \item Was ist der Name des Biologen, mit dem du gekommen bist?
    \item Der Beruf des Polizisten ist sehr anspruchsvoll.
    \item Man soll mit dem Idioten nicht reden.
    \item Die Farbe des Herzens ist rot.
    \item Die Stadt Koblenz machte dem Kronprinzen Friedrich Wilhelm von Preu{\ss}en das Schloss Stolzenfels zum Geschenk.
    \item Die Zuverl\"assigkeit der Deutschen ist der Grund ihres Erfolgs.
    \item Wenn du den Frieden willst, bereite den Krieg vor. (Si vis pacem para bellum)
    \item Heiko kam zur Party mit seinem Affen.
    \item Die Anforderungen der Kunden sind manchmal unpr\"azis.
\end{enumerate}
\end{document}
