\documentclass[a4paper,12pt]{article}

%\title{Deutsch: Die Hausaufgaben}
%\date{Oktober 15, 2014}
%\author{D.Babula \\ Bury GmBH L\"ohne Deutschlang }
\makeatletter

\def\thickhrulefill{\leavevmode \leaders \hrule height 1pt\hfill \kern \z@}
\def\maketitle{%
    \null
    \thispagestyle{empty}%
    \vfill
    \begin{center}\leavevmode
    \normalfont
    {\LARGE\raggedleft \@author\par}%
    \thickhrulefill\par
    {\huge\raggedright \@title\par}%
    \vskip 1cm
    \end{center}%
    \vfill
    {\Large \@date\par}%
    \null
    \cleardoublepage
    }
    \makeatother
    \author{Dawid Babula}
    \title{Deutsch -- die Hausaufgaben}
    \date{Oktober 15, 2014}

\begin{document}
\maketitle

\textbf{\large N-Deklination}
\vspace{2pc}
\begin{enumerate}
    \item K\"onnten Sie mir bitte den Namen Ihrer Frau buchstabieren?
    \item Der Name des ersten amerikanischen Pr\"asidenten war George.
    \item Die Gastfreundschaft der Polen ist weltweit ber\"umt.
    \item Heute,kommen meine Mutter mit meinem Neffen bei uns vorbei.
    \item Der Gr\"undungsmythos Krakaus ist die Sage von dem Krak,der den Drachen get\"otet hatte.
    \item Ich suche nach dem guten Blumenlieferanten,\"uber dem ich meiner Frau einen sch\"onen Strau{\ss} schicke.
    \item Was ist der Name des Biologen, mit dem du gekommen bist?
    \item Der Beruf des Polizisten ist sehr anspruchsvoll.
    \item Man soll mit dem Idioten nicht reden.
    \item Die Farbe des Herzens ist rot.
    \item Die Stadt Koblenz machte dem Kronprinzen Friedrich Wilhelm von Preu{\ss}en das Schloss Stolzenfels zum Geschenk.
    \item Die Zuverl\"assigkeit der Deutschen ist der Grund ihres Erfolgs.
    \item Wenn du den Frieden willst, bereite den Krieg vor. (Si vis pacem para bellum)
    \item Heiko kam zur Party mit seinem Affen.
    \item Die Anforderungen der Kunden sind manchmal unpr\"azis.
\end{enumerate}
\newpage
\textbf{\large Was w\"urderst du machen, wenn du viel Zeit h\"attest ?}
\vspace{2pc}
\\{}
Meiner Meinung nach ist, die Antwort auf die Titelfrage ist nicht so einfach. Manchmal denken wir \"uber die unendliche Lebenszeit nicht nach, weil wir alle sterblich sind.
Vor einem Jahr habe ich einen Film angeschaut, er hei{\ss}t "Gespr\"ach mit einem Vampir". Der Film erz\"alt \"uber das Leben des Vampirs Louis. Der Vampir Louis ist
c.a. 200 hundert Jahre alt. Das bedeutet, dass er zweimal mehr Zeit wie ein normaller Mensch hat. Was interesant ist ,dass er gelangweilt vom Leben erscheint.
Er hat durch sein Leben so viele Sachen erfahren, so viele Leute kennengelernt, trotz dem ist er gelangweilt. Ich glaube,dass das w\"urde mit jedem Lebewesen passieren, egal ob
er ein Mensch ist oder ein Vampir. Nat\"urlich, denke ich jetzt,dass ich ein so langes Leben schaffen w\"urde, aber die Wirklichkeit kann ein bisschen anders sein.
Am Anfang w\"urde ich ein eigene Wohnung kaufen, dann w\"urde ich mehr Fremdsprachen lernen. Es w\"are gut, wenn andere Leute auch so lang leben k\"onnten. Besonders, meine ich meine
Frau. Ich glaube, dass man nicht allein leben soll, weil man ungl\"ucklich wird.

\end{document}
