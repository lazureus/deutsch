%%%%%%%%%%%%%%%%%%%%%%%%%%%%%%%%%%%%%%%%%
% Journal Article
% LaTeX Template
% Version 1.3 (9/9/13)
%
% This template has been downloaded from:
% http://www.LaTeXTemplates.com
%
% Original author:
% Frits Wenneker (http://www.howtotex.com)
%
% License:
% CC BY-NC-SA 3.0 (http://creativecommons.org/licenses/by-nc-sa/3.0/)
%
%%%%%%%%%%%%%%%%%%%%%%%%%%%%%%%%%%%%%%%%%

%----------------------------------------------------------------------------------------
%	PACKAGES AND OTHER DOCUMENT CONFIGURATIONS
%----------------------------------------------------------------------------------------

\documentclass[twoside]{article}

\usepackage{lipsum} % Package to generate dummy text throughout this template

\usepackage[sc]{mathpazo} % Use the Palatino font
\usepackage[T1]{fontenc} % Use 8-bit encoding that has 256 glyphs
\linespread{1.05} % Line spacing - Palatino needs more space between lines
\usepackage{microtype} % Slightly tweak font spacing for aesthetics

\usepackage[hmarginratio=1:1,top=32mm,columnsep=20pt]{geometry} % Document margins
\usepackage{multicol} % Used for the two-column layout of the document
\usepackage[hang, small,labelfont=bf,up,textfont=it,up]{caption} % Custom captions under/above floats in tables or figures
\usepackage{booktabs} % Horizontal rules in tables
\usepackage{float} % Required for tables and figures in the multi-column environment - they need to be placed in specific locations with the [H] (e.g. \begin{table}[H])
\usepackage{hyperref} % For hyperlinks in the PDF

\usepackage{lettrine} % The lettrine is the first enlarged letter at the beginning of the text
\usepackage{paralist} % Used for the compactitem environment which makes bullet points with less space between them

\usepackage{abstract} % Allows abstract customization
\renewcommand{\abstractnamefont}{\normalfont\bfseries} % Set the "Abstract" text to bold
\renewcommand{\abstracttextfont}{\normalfont\small\itshape} % Set the abstract itself to small italic text

\usepackage{titlesec} % Allows customization of titles
\renewcommand\thesection{\Roman{section}} % Roman numerals for the sections
\renewcommand\thesubsection{\Roman{subsection}} % Roman numerals for subsections
\titleformat{\section}[block]{\large\scshape\centering}{\thesection.}{1em}{} % Change the look of the section titles
\titleformat{\subsection}[block]{\large}{\thesubsection.}{1em}{} % Change the look of the section titles

\usepackage{fancyhdr} % Headers and footers
\pagestyle{fancy} % All pages have headers and footers
\fancyhead{} % Blank out the default header
\fancyfoot{} % Blank out the default footer
\fancyhead[C]{ Wenn W\"orter auf die Reise gehen $\bullet$ April 2015 } % Custom header text
\fancyfoot[RO,LE]{\thepage} % Custom footer text

%----------------------------------------------------------------------------------------
%	TITLE SECTION
%----------------------------------------------------------------------------------------

\title{\vspace{-15mm}\fontsize{24pt}{10pt}\selectfont\textbf{Wenn W\"orter auf die Reise gehen}} % Article title

\author{
\large
\textsc{Unknown Author}\\[2mm] % Your name
\normalsize Neue Westf\"alische \\ % Your institution
\normalsize \href{mailto:gfsekretariat@nw.de}{gfsekretariat@nw.de} % Your email address
\vspace{-5mm}
}
\date{}

%----------------------------------------------------------------------------------------

\begin{document}

\maketitle % Insert title

\thispagestyle{fancy} % All pages have headers and footers

%----------------------------------------------------------------------------------------
%	ABSTRACT
%----------------------------------------------------------------------------------------

\begin{abstract}

\begin{center}
\noindent Philipp Reis entwickelte vor 150 Jahren das erste Telefon
\end{center}

\end{abstract}

%----------------------------------------------------------------------------------------
%	ARTICLE CONTENTS
%----------------------------------------------------------------------------------------

\begin{multicols}{2} % Two-column layout throughout the main article text

\section{DER TEST}

\lettrine[nindent=0em,lines=3]{D}as Pferd frisst\footnote{fressen - fra{\ss} - hat gefressen - }
keinen Gurkensalat\footnote{der Salat - sa\l{}atka}.Und: "Die Sonne ist von Kupfer\footnote{das Kupfer - mied\'z}." Was ist denn das f\"ur ein Quatsch?
 Quatsch ist ein gutes Stichwort. Besser gesagt: quatschen.
 Denn diese S\"atze wurden vor vielen Jahren gesprochen,
 um ein besonderes, neues Ger\"at zu testen: das Telefon.
 So nannte der deutsche Lehrer Philipp Reis seine Erfindung.
 Er hatte lange Zeit daran herumget\"uftelt, bis er schlie{\ss}lich
 vor rund 150 Jahren sein Telefon das erste Mal testete.

%------------------------------------------------

\section{ELEKTRISCHE SIGNALE}

Das Telefon bestand aus einer Art h\"olzernen Ohrmuschel. Dar\"uber
 spannte Philipp Reis ein St\"uck Tierhaut. Zudem geh\"orten zu seinem
 Telefon unter anderem noch eine Batterie und eine Stricknadel. Klingt
 ziemlich verr\"uckt, aber man konnte damit tats\"achlich W\"orter
 auf die Reise schicken. Die Konstruktion von Philipp Reis verwandelte
 das Gesprochene in elektrische Signale. Das ging dann durch einen
 Draht und wurde am anderen Ende wieder zu gesprochenen Worten.

%------------------------------------------------

\section{ERFINDUNG OHNE ERFOLG}

Allerdings funktionierte das Ganze noch nicht einwandfrei. Als sein
 Kollege den Satz mit der Sonne aus Kupfer sprach,verstand der Efinder
 wohl: "Die Sonne ist von Zucker." Philipp Reis entwickelte sein Telefon
 noch weiter. Gro{\ss}en Erfolg hatte er aber nicht. Das war bei Alexander
 Graham Bell einige Jahre sp\"ater anders. Er wird heute oft als Erfinder
 des Telefons genannt, denn er bekam das Patent f\"ur die Erfindung.Er wurde
 als Erfinder anerkannt.

%------------------------------------------------

\end{multicols}

\end{document}
