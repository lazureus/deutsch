%%%%%%%%%%%%%%%%%%%%%%%%%%%%%%%%%%%%%%%%%
% Journal Article
% LaTeX Template
% Version 1.3 (9/9/13)
%
% This template has been downloaded from:
% http://www.LaTeXTemplates.com
%
% Original author:
% Frits Wenneker (http://www.howtotex.com)
%
% License:
% CC BY-NC-SA 3.0 (http://creativecommons.org/licenses/by-nc-sa/3.0/)
%
%%%%%%%%%%%%%%%%%%%%%%%%%%%%%%%%%%%%%%%%%

%----------------------------------------------------------------------------------------
%	PACKAGES AND OTHER DOCUMENT CONFIGURATIONS
%----------------------------------------------------------------------------------------

\documentclass[twoside]{article}

\usepackage{lipsum} % Package to generate dummy text throughout this template

\usepackage[sc]{mathpazo} % Use the Palatino font
\usepackage[T1]{fontenc} % Use 8-bit encoding that has 256 glyphs
\linespread{1.05} % Line spacing - Palatino needs more space between lines
\usepackage{microtype} % Slightly tweak font spacing for aesthetics

\usepackage[hmarginratio=1:1,top=32mm,columnsep=20pt]{geometry} % Document margins
\usepackage{multicol} % Used for the two-column layout of the document
\usepackage[hang, small,labelfont=bf,up,textfont=it,up]{caption} % Custom captions under/above floats in tables or figures
\usepackage{booktabs} % Horizontal rules in tables
\usepackage{float} % Required for tables and figures in the multi-column environment - they need to be placed in specific locations with the [H] (e.g. \begin{table}[H])
\usepackage{hyperref} % For hyperlinks in the PDF

\usepackage{lettrine} % The lettrine is the first enlarged letter at the beginning of the text
\usepackage{paralist} % Used for the compactitem environment which makes bullet points with less space between them

\usepackage{abstract} % Allows abstract customization
\renewcommand{\abstractnamefont}{\normalfont\bfseries} % Set the "Abstract" text to bold
\renewcommand{\abstracttextfont}{\normalfont\small\itshape} % Set the abstract itself to small italic text

\usepackage{titlesec} % Allows customization of titles
\renewcommand\thesection{\Roman{section}} % Roman numerals for the sections
\renewcommand\thesubsection{\Roman{subsection}} % Roman numerals for subsections
\titleformat{\section}[block]{\large\scshape\centering}{\thesection.}{1em}{} % Change the look of the section titles
\titleformat{\subsection}[block]{\large}{\thesubsection.}{1em}{} % Change the look of the section titles

\usepackage{fancyhdr} % Headers and footers
\pagestyle{fancy} % All pages have headers and footers
\fancyhead{} % Blank out the default header
\fancyfoot{} % Blank out the default footer
\fancyhead[C]{ Unknown $\bullet$ May 2015 } % Custom header text
\fancyfoot[RO,LE]{\thepage} % Custom footer text

%----------------------------------------------------------------------------------------
%	TITLE SECTION
%----------------------------------------------------------------------------------------

\title{\vspace{-15mm}\fontsize{24pt}{10pt}\selectfont\textbf{Unknown}} % Article title

\author{
\large
\textsc{Unknown Author}\\[2mm] % Your name
\normalsize Unknown \\ % Your institution
\normalsize \href{mailto:gfsekretariat@nw.de}{gfsekretariat@nw.de} % Your email address
\vspace{-5mm}
}
\date{}

%----------------------------------------------------------------------------------------

\begin{document}

\maketitle % Insert title

\thispagestyle{fancy} % All pages have headers and footers

%----------------------------------------------------------------------------------------
%	ABSTRACT
%----------------------------------------------------------------------------------------

\begin{abstract}

\begin{center}
    \noindent
\end{center}

\end{abstract}

%----------------------------------------------------------------------------------------
%	ARTICLE CONTENTS
%----------------------------------------------------------------------------------------

\begin{multicols}{2} % Two-column layout throughout the main article text

\section{TBD}

\lettrine[nindent=0em,lines=3]{H}erzog stammt\footnote{stammen - stammte - hat
 gestammt -- pochodzi\'c} aus einer Bergarbeiterfamilie,hat selbst 25 Jahre im Bergbau gearbeitet, ist heute Bauleiter bei einem
 Verkehrssicherungsunternehmen. Und nebenbei Guide. ,,Mensch, ist das gr\"un hier``,
 sei das Erste, was Besucher sagen, wenn sie von den H\"ugeln\footnote {der H\"ugel -- pag\'{o}rek}
 aufs Ruhrgebiet hinabschauen.\footnote{hinabschauen - schaute hinab - hat hinabgeschaut -- spogl\k{a}da\'c w d\'{o}\l{}}
 Ja, es ist tats\"achlich erstaunlich gr\"un. Allerdings\footnote{allerdings - jednak,wprawdzie}
 r\"ucken \footnote{r\"ucken - r\"uckte - hat ger\"uckt -- przesuwa\'c si\k{e}, znajdowa\'c si\k{e}}
 auch deutlich mehr Kraftwerke, M\"ullverbrennungsanlagen, Hallend\"acher und L\"armschutzw\"alle
 in den Blick\footnote{in den Blick r\"ucken - pojawia\'c si\k{e} w zasi\k{e}gu wzroku} , als man das aus klassischen Bergregionen kennt. Nur von Freizeit
 k\"onne man eben nicht leben, gibt Herzog zu bedenken.\footnote{bedenken - bedachte - hat bedacht -- zastanawia\'c sie\k{e}, przemy\'sle\'c}
 \footnote{zu bedenken geben, dass ... -- zwaraca\'c komu\'s uwag\k{e} \.{z}e} Das Land der rauchenden Schlote\footnote{der Schlot -- komin}
 aber sei das hier l\"angst nicht mehr. Zwei Zechen in Bottrop und Marl bauen noch ab,
 die letzte schlie{\ss}t 2018. Dann ist der Bergbau im Ruhrgebiet Geschichte.,,Hier
 wird keine W\"asche mehr schwarz beim Raush\"angen``\footnote{raush\"angen - hing raus - hat rausgehangen -- wywiesza\'c}, betont
 \footnote{betonen - betonte - hat betonnt - podkre\'sla\'c}Herzog. Das schlechte
 Image halte sich dennoch hartn\"ackig.\footnote{hartn\"ackig -- zawzi\k{e}ty, uparty} Auch hier soll HHH helfen: der gr\"une Pott --
 das ist das Bild, das Sven Ahrens in die K\"opfe pflanzen m\"ochte. Der Weg f\"uhrt
 jetzt bergan\footnote{bergan -- do g\'ory}, ein steiniger\footnote{steinig - kamienisty} ist es nicht, stattdessen ist er sauber gepflastert.
 \footnote{pflastern - pflasterte - hat gepflastert -- brukowa\'c, wyk\l{}ada\'c}
 Ohnehin\footnote{ohnehin -- tak czy owak} sind 51 Prozent der f\"urs HHH genutzten Wege asphaltiert\footnote{asphaltieren - asphaltierte - hat asphaltiert -- asfaltowa\'c}
 ,fast 45 Prozent mit Schotter\footnote{der Schotter -- \.zwir,szuter} und Schlacke\footnote{die Schlacke -- \.zu\.zel} bedeckt, nur f\"unf Prozent Naturboden.
 Erdklumpen\footnote{der Erdklumpen -- bry\l{}ka ziemi} muss man sich nach der Wanderung keine aus dem Profil\footnote{das Profil -- protektor,podeszwa}
 pulen.\footnote{pulen - pulte - hat gepult -- obiera\'c}
 Neben der Bewegung an der frischen Luft geht es bein Halden-H\"ugen-Hopping auch um Inhalte.,,Wenn man Industriekultur
 erwandert\footnote{erwandern - erwanderte - hat erwandert - poznawa\'c pieszo}, stellt man Zusammenh\"ange\footnote{der Zusammenhang -- kontekst}
 nicht alleine her``, sagt der Ex-Bergmann Herzog.
 Vieles ist erkl\"arungsbed\"urftig.\footnote{erkl\"arungsbed\"urftig -- wymagaj\k{a}cy wyja\'snienia} An 150 festgelegten Erz\"ahlstationen soll beim HHH Wissen
 vermittelt\footnote{vermitteln - vermittelte - hat vermittelt -- przekazywa\'c} werden -- durch einen pers\"onlichen
 Guide oder \"ubers Halden-H\"ugel-Navi, eine App f\"urs Smartphone. Leicht au{\ss}er Puste\footnote{die Puste/- -- dech}\footnote{au{\ss}er Puste sein -- nie m\'oc
 z\l{}apa\'c tchu} erreichen wir das Haldentop, wo nicht etwa ein simples\footnote{simpel -- prosty, zwyk\l{}y}
 Gipfelkreuz auf uns wartet, sondern zwei gegeneinander verschobene\footnote{verschieben - verschob - hat verschoben -- przesuwa\'c} Stahlrohrb\"ogen mit einem
 Radius von 45 Metern. Die astronomische Kenntnisse, die dieses Horizont-Observatorium vermittelt soll, bleiben uns aber verborgen,\footnote{verbergen - verbarg - hat verborgen
 -- ukrywa\'c, zataja\'c} denn ein Bauzaun\footnote{der Bauzaun -- ogrodzenie budowy} schirmt es ab.\footnote{abschirmen - schirmte ab - hat abgeschirmt -- os\l{}ania\'c,
 ochrania\'c} Kurz nach der Er\"offnung.

%------------------------------------------------

 \section{Eine Jausenstation hier oben w\"are jetzt was.Landestypisch m\"usste es eine
 Trinkhalle sein}

 Ende 2008 sprang eine Schwei{\ss}naht\footnote{die Schwei{\ss}naht -- spaw, spoina} auf.\footnote{aufspringen - sprang auf - ist aufgesprungen -- p\k{e}ka\'c}
 Seither\footnote{seither -- od tego czasu}  streitet man, wessen Schuld das ist. Das Observatorium gilt als\footnote{als etw. gelten -- co\'s ma jako\'s opinie}
 eine der neuen Landmarken im Ruhrgebiet, gemeinsam mit dem Gasometer\footnote{der Gasometer -- ogromny zbiornik gazu} in Oberhausen und dem Tetraeder
 \footnote{das Tetraeder -- czworo\'scian foremny} in Bottrop. L\"anst \"uberragen\footnote{\"uberragen - \"uberragte - hat \"uberragt -- przewy\.zsza\'c, wznosi\'c si\k{e} ponad
 czym\'s} sie die verbliebenen\footnote{verbleiben - verblieb - ist verblieben -- pozostawa\'c} F\"odert\"urme\footnote{der F\'orderturm -- wie\.za wyci\k{a}gowa}, an denen man sich
 fr\"uher orientierte. Hundert Meter \"uber Grund klingt wenig, relativiert\footnote{relativieren - relativierte - hat relativiert -- relatywizowa\'c} sich hier oben aber schnell.
 Man steht \"uber allem. Selbst der Verkehrsl\"arm\footnote{der Verkehrsl\"arm -- ha\l{}as komunikacyjny}
 ist weitestgehend\footnote{weitgehend -- daleko id\k{a}cy} verschwunden. ,,Anfangs wurder wir ja bel\"achelt\footnote{bel\"acheln - bel\"achelte - hat bel\"achelt -- pod\'smiechiwa\'c
 si\k{e}} f\"ur die Sache mit dem Wandern``, sagt Markus Keil, der Leiter des Besucherzentrums
 Hoheward. Dabei habe es Potenzial f\"ur die ganze Metropole Ruhr. Er schw\"armt\footnote{schw\"armen - schw\"armte - hat geschw\"armt}
 von den urbanen Panoramen, etwa von n\"achtlichen Bild des Horizont-Observatoriums
 \"uber dem Lichtermeer von Recklinghausen, das ihn jedes Mal an das Griffith Observatory auf den Hollywood Hills erinnert, wo einem L.A. zu F\"u{\ss}en liegt.
 Vielleicht fehlt dem Pott ja nicht das Faszinierende, sondern das Selbstbewusstsein. Eine Jausenstation\footnote{die Jausenstation -- restauracja w obszarze wycieczkowym}
 hier oben, das w\"are jetzt was. Landestypisch m\"usste das dann eine Trinkhalle\footnote{die Trinkhalle -- pijalnia ,kiosk z napojami}sein.
 Die liegen nat\"urlich unten in den Wohngebieten. Oder fr\"uher zwischen Zechentor und Zuhause, dort kamen die Bergleute vorbei
 und hatten m\"achtig Durst. Auch an den Hopping-Routen stehen sie. Die Route HN am Rande der Haard etwa hat die Trinkhallenkultur eigens als Thema im Programm.
 Unterwegs macht man Stopp an der ,,Ballerbude``, an der sich auch der \"ortliche Fu{\ss}ballverein trifft. Im Landschaftspark Hoheward ist ein kurzer
 Abstecher\footnote{der Abstecher -- objazd} n\"otig, um ,,Bei Stefan`` in Wanne einzukehren.\footnote{einkehren - kehrte ein - ist eingekehrt - zago\'sci\'c, wst\k{e}powa\'c}
 Eine Familie holt sich gerade Eis, die Mutter r\"umpft\footnote{r\"umpfen - r\"umpfte - hat ger\"umpft -- kr\k{e}ci\'c nosem} belustigt\footnote{belustigt -- weso\l{}y, zabawny}
 die Nase auf die Frage, ob sie auch gern auf die Halden steige. ,,Auf unsere M\"ullalpen?``, erwidert\footnote{erwidern - erwiderte - hat erwidert -- odrzec,odwzajemnia\'c} sie.
 Wenn sie wandern wolle, fahre sie in ein Wandergebiet. Aber vielleicht wird ja ein Austausch
 daraus: Die Leute aus dem Schwarzwald kommen zum Wandern ins Ruhrgebiet und umgekehrt. Denn schlussendlich\footnote{schlussendlich -- ostatecznie, ewentualnie}
 reizt\footnote{reizen - reizte - hat gereizt -- ciekawi\'c, prowokowa\'c,n\k{e}ci\'c} doch immer das Unbekannte - ganz gleich, ob das nat\"urlich oder k\"nstlich ist.

%------------------------------------------------

\end{multicols}

\end{document}
