%%%%%%%%%%%%%%%%%%%%%%%%%%%%%%%%%%%%%%%%%
% Journal Article
% LaTeX Template
% Version 1.3 (9/9/13)
%
% This template has been downloaded from:
% http://www.LaTeXTemplates.com
%
% Original author:
% Frits Wenneker (http://www.howtotex.com)
%
% License:
% CC BY-NC-SA 3.0 (http://creativecommons.org/licenses/by-nc-sa/3.0/)
%
%%%%%%%%%%%%%%%%%%%%%%%%%%%%%%%%%%%%%%%%%

%----------------------------------------------------------------------------------------
%	PACKAGES AND OTHER DOCUMENT CONFIGURATIONS
%----------------------------------------------------------------------------------------

\documentclass[twoside]{article}

\usepackage{lipsum} % Package to generate dummy text throughout this template

\usepackage[sc]{mathpazo} % Use the Palatino font
\usepackage[T1]{fontenc} % Use 8-bit encoding that has 256 glyphs
\linespread{1.05} % Line spacing - Palatino needs more space between lines
\usepackage{microtype} % Slightly tweak font spacing for aesthetics

\usepackage[hmarginratio=1:1,top=32mm,columnsep=20pt]{geometry} % Document margins
\usepackage{multicol} % Used for the two-column layout of the document
\usepackage[hang, small,labelfont=bf,up,textfont=it,up]{caption} % Custom captions under/above floats in tables or figures
\usepackage{booktabs} % Horizontal rules in tables
\usepackage{float} % Required for tables and figures in the multi-column environment - they need to be placed in specific locations with the [H] (e.g. \begin{table}[H])
\usepackage{hyperref} % For hyperlinks in the PDF

\usepackage{lettrine} % The lettrine is the first enlarged letter at the beginning of the text
\usepackage{paralist} % Used for the compactitem environment which makes bullet points with less space between them

\usepackage{abstract} % Allows abstract customization
\renewcommand{\abstractnamefont}{\normalfont\bfseries} % Set the "Abstract" text to bold
\renewcommand{\abstracttextfont}{\normalfont\small\itshape} % Set the abstract itself to small italic text

\usepackage{titlesec} % Allows customization of titles
\renewcommand\thesection{\Roman{section}} % Roman numerals for the sections
\renewcommand\thesubsection{\Roman{subsection}} % Roman numerals for subsections
\titleformat{\section}[block]{\large\scshape\centering}{\thesection.}{1em}{} % Change the look of the section titles
\titleformat{\subsection}[block]{\large}{\thesubsection.}{1em}{} % Change the look of the section titles

\usepackage{fancyhdr} % Headers and footers
\pagestyle{fancy} % All pages have headers and footers
\fancyhead{} % Blank out the default header
\fancyfoot{} % Blank out the default footer
\fancyhead[C]{ Herrscherin \"uber Russland $\bullet$ May 2015 } % Custom header text
\fancyfoot[RO,LE]{\thepage} % Custom footer text

%----------------------------------------------------------------------------------------
%	TITLE SECTION
%----------------------------------------------------------------------------------------

\title{\vspace{-15mm}\fontsize{24pt}{10pt}\selectfont\textbf{Herrscherin \"uber Russland}} % Article title

\author{
\large
\textsc{Unknown Author}\\[2mm] % Your name
\normalsize Neue Westf\"alische \\ % Your institution
\normalsize \href{mailto:gfsekretariat@nw.de}{gfsekretariat@nw.de} % Your email address
\vspace{-5mm}
}
\date{}

%----------------------------------------------------------------------------------------

\begin{document}

\maketitle % Insert title

\thispagestyle{fancy} % All pages have headers and footers

%----------------------------------------------------------------------------------------
%	ABSTRACT
%----------------------------------------------------------------------------------------

\begin{abstract}

\begin{center}
    \noindent Beliebt beim Volk: Katharina die Gro{\ss}e regierte 34 Jahre das Russische Kaiserreich
    als Zarin. Das war dort der h\"ochste Herrschaftstitel. Es gibt viele Gem\"alde und Denkm\"aler,
    die Katherina zeigen. Katharina die Gro{\ss}e war eine der m\"achtigsten Frauen der Welt.
\end{center}

\end{abstract}

%----------------------------------------------------------------------------------------
%	ARTICLE CONTENTS
%----------------------------------------------------------------------------------------

\begin{multicols}{2} % Two-column layout throughout the main article text

\section{ENTT\"AUSCHTE MUTTER}

\lettrine[nindent=0em,lines=3]{E}in M\"adchen? Die Mutter der kleinen Prinzessin war bei der
 Geburt ihrer Tochter am 2.Mai 1729 entt\"auscht. Sie hatte sich einen Sohn gew\"unscht. Dass
 aus dem M\"adchen einmal Katharina die Gro{\ss}e , Herrscherin \"uber das riesige Russland,
 werden w\"urde, ahnte die Mutter nicht, im Gegenteil. Katharina hie{\ss} zuerst anders:
 Sophie Auguste Friederike Prinzessin von Anhalt-Zerbst liegt im heutigen Bundesland Sachsen-Anhalt.
%------------------------------------------------

\section{VERLOBT MIT 14 JAHREN}

F\"ur adelige M\"adchen wie sie war in der Regel eine bestimmte Rolle vorgesehen: Sie sollten
 in eine Adelsfamilie einheiraten, ihren Mann unterst\"utzen und Kinder bekommen. Anfangs
 sah es so aus, als gelte das auch f\"ur Sophie. Als sie 14 Jahre alt war, wurde vereinbart,
 dass sie den Neffen der russischen Zarin heiraten, den sp\"ateren Zaren Peter III. Sophie
 kannte der Mann kaum. Aber sp\"ater schrieb sie, er sei ihr egal gewesen. Sie habe es
 aufregend gefunden.
%------------------------------------------------

\section{SOPHIE WIRD KATHARINA}

Sophie lernte Russisch und nahm einen russischen Namen an:
 Aus Sophie wurde Katharina. Einige Jahre sp\"ater herrschte
 Peter als Kaiser. Er wollte aber eine andere Frau haben
 und Katharina loswerden. Da verk\"undete Katharina dem Volk,
 sie selbst sei jetzt die Kaiserin. Sie war sehr beliebt,
 Peter nicht. Das Volk jubelte,und Peter musste Katharina die
 Krone \"uberlassen. Sie regierte 34 Jahre lang und erhielt
 als erste Frau den Beinamen "die Gro{\ss}e".

%------------------------------------------------

\end{multicols}

\end{document}
